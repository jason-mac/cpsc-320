\section{Very Special Problems}

In this question, we consider special cases of NP-complete problems. Formally, we say that Problem B is a \textbf{special case} of Problem A if every instance of Problem B can be viewed as an instance of Problem A. We've seen examples of this in class: for example, 3-SAT is a special case of SAT (where every clause has length 3). Minimum spanning tree is a special case of Steiner Tree, in which the set of vertices we need to connect includes all the vertices in $V$.

\begin{questions}

	\question[4] Consider the \textbf{Bounded-Leaf Spanning Tree Problem (BLST)}: given a graph $G = (V,E)$ and an integer $k$, does $G$ have a spanning tree with no more than $k$ leaves?

	Give an NP-complete problem that is a special case of BLST, and justify why this problem is a special case.

	\ifsolutions\begin{soln}
	Let \(V\) be a vertex set. Add \(v \in V\). Then also add \(v_1, v_2, \dots, v_n\) to \(V\).

	Then for \(v_1, v_2, \dots, v_d\) add \((v, v_i) \in E\) for \(i = 1, 2, \dots, d\).

	Then colour the node \(v\) blue, and colour all other nodes in \(V\) red.

	By construction, the degree of all other nodes that are not \(v\) is \(1\) since it's only connected to \(v\).

	Thus, the degree of \(v\) is the maxmimum degree of the graph, which is \(d\).

	Since no nodes are adjacent except for ones to \(v\). Then the colour red is never shared by adjacent vertices.

	Thus, this graph with maximum degree \(d\) can be coloured with two colours.

\end{soln}
\fi

	\begin{soln}
		Let \(G = (V, E)\) be a graph. Suppose that this graph had a hamiltonian path.

		This means we can form a spanning tree out of this graph using exactly \(n - 1\) edges.

		Let \(P = (v_1, v_2, \dots, v_n)\) be the hamiltonian path in \(G\), and consider it the spanning tree structure.

		By definition, a leaf in a tree is a node with degree \(1\). We see that \(v_1, v_n\) both have degree \(1\).

		Hence, this tree has exactly two leaves.

		Thus, if we take \(G\) and ask the question to HAMPATH, does this graph contain a hamiltonian path?

		If the reply is yes, then \(G\) has a spanning tree with \(2\) leaves, or we can also say no more than \(2\) leaves.

		In other words, HAMPATH is a special case of BLST where we specify \(k = 2\).

	\end{soln}

	\question[3] You showed in the previous question that there is an NP-complete problem that is a special case of BLST, and it is not difficult to show that BLST is in NP (though we are not asking you to do this). Does this imply that BLST is NP-complete? Justify your answer.

	\ifsolutions\begin{soln}
	Proof: We provide a proof by induction on the number of nodes, \(n \in \mathbb{N}\), for graph \(G = (V, E)\).

	Base case \(|V| = 1\). The highest degree can be \(d = 0\), since the edge set would be empty.

	Thus, the algorithm uses \(1 = d + 1\) colours to colour the single node and the base case holds.

	Assume for any graph with \(n\) nodes that the algorithm will colour any ordering of nodes using at most \(d + 1\) colours.

	Consider a graph \(G = (V, E)\) with \(|V| = n + 1\) nodes with the highest degree of any node is \(d\).

	Let \(v_1, v_2, \dots, v_n, v_{n+1}\) be any ordering of the nodes. Remove \(v_{n+1}\) from this order, and the graph.

	Denote the deleted graph without \(v_{n+1}\) by \(G'\). Notice the degree of any node in \(G'\) can only decrease.

	Thus, the maximum degree of \(G'\) remains to be \(d\) with the removal of \(v_{n+1}\).

	By assumption, we can colour the ordering \(v_1, v_2, \dots, v_n\) using at most \(d+1\) colours.

	Since \(v_{n+1}\) has degree at most \(d\) then it has at most \(d\) neighbours. Then we add back \(v_{n+1}\).

	The number of colours used to colour its neigbours can be at most \(d\), if each are coloured distinctly.

	Thus, there remains at least \(1\) unique colour to colour \(v_{n+1}\) so that it is a valid colouring.

	In other words, the algorithm uses at most \(d + 1\) colours to colour the ordering \(v_1, v_2, \dots, v_{n+1}\).

	Induction makes the claim holds true for any graph with \(n\) nodes and maximum degree \(d\).


\end{soln}
\fi

	\begin{soln}
		This does not necessarily imply that BLST is NP-complete. Let \(Y\) be some NP-complete problem.

		By definition, to prove BLST is NP-complete, it remains to show that \(Y \leq_p BLST\).

		While we showed in the previous part that an NP-complete problem (HAMPATH) is a special case of BLST (for \( k = 2 \)), this is not sufficient to conclude NP-completeness of BLST in general.

		In particular, it remains to show that \(BLST(k = 2) \leq_p BLST\) or equivalently \(HAMPATH \leq_p BLST\).

		That is, we must show a polynomial time reduction from HAMPATH (or any another problem \(Y\)) to BLST, not just observe that one is a special case of the other.

		Hence, unless such a reduction is explicitly given with proof of correctness, we cannot conclude that BLST is NP-complete.
	\end{soln}


	\question[4] Give an example of a polytime-solvable problem you have seen in this class that is a special case of an NP-complete problem. Justify your answer.

	\ifsolutions\begin{soln}
	We consider the graph \(G = (V, E)\) with \(V = \{1, 2, 3, 4\}\) and \(E = \{(1, 4), (4, 3), (3, 2)\}\).

	\begin{center}
		\begin{tikzpicture}[node distance=2cm, auto]
			% Nodes
			\node[circle, draw, above  right = of 3] (1) {1};
			\node[circle, draw, right=of 1] (2) {2};
			\node[circle, draw, right=of 2] (3) {3};
			\node[circle, draw, above right=of 2] (4) {4};

			% Edges
			\draw (1) -- (4) -- (3) -- (2);
		\end{tikzpicture}
	\end{center}

	This graph can be coloured with three colours. Namely we can assign \(A = \{1, 3\}\) \(B = \{2, 4\}\).

	We see that no edges are shared between any vertices in \(A, B\) so this is a valid colouring.

	Now consider the ordering \(1, 2, 3, 4\). We first colour \(1\) blue. And then we consider \(2\).

	There is no edge between \(1, 2\) so we colour \(2\) blue and consider \(3\).

	So, its adjacent vertices have been coloured with blue, then we introduce a new colour for it red.

	Now we finally consider \(4\), its neighbours have been coloured with both red and green, so we must introduce a new colour for it purple.

	We have thus coloured this graph using three colours through the greedy algorithm when we could have used two.


\end{soln}
\fi

	\begin{soln}
		Recall the problem of finding a topological ordering on a directed graph \(G = (V, E)\).

		The dynamic programming solution can produce a solution in polynomial time.



		Now recall the
	\end{soln}

\end{questions}
