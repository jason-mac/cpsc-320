\section{Very Special Problems}

In this question, we consider special cases of NP-complete problems. Formally, we say that Problem B is a \textbf{special case} of Problem A if every instance of Problem B can be viewed as an instance of Problem A. We've seen examples of this in class: for example, 3-SAT is a special case of SAT (where every clause has length 3). Minimum spanning tree is a special case of Steiner Tree, in which the set of vertices we need to connect includes all the vertices in $V$.

\begin{questions}

	\question[4] Consider the \textbf{Bounded-Leaf Spanning Tree Problem (BLST)}: given a graph $G = (V,E)$ and an integer $k$, does $G$ have a spanning tree with no more than $k$ leaves?

	Give an NP-complete problem that is a special case of BLST, and justify why this problem is a special case.

	\ifsolutions\begin{soln}
	Let \(V\) be a vertex set. Add \(v \in V\). Then also add \(v_1, v_2, \dots, v_n\) to \(V\).

	Then for \(v_1, v_2, \dots, v_d\) add \((v, v_i) \in E\) for \(i = 1, 2, \dots, d\).

	Then colour the node \(v\) blue, and colour all other nodes in \(V\) red.

	By construction, the degree of all other nodes that are not \(v\) is \(1\) since it's only connected to \(v\).

	Thus, the degree of \(v\) is the maxmimum degree of the graph, which is \(d\).

	Since no nodes are adjacent except for ones to \(v\). Then the colour red is never shared by adjacent vertices.

	Thus, this graph with maximum degree \(d\) can be coloured with two colours.

\end{soln}
\fi

	\begin{soln}
		Let an instance of BLST be given where \(k\) is specified to be \(2\) and that the answer is yes.

		Then we have some spanning tree \(T\) of \(G\) such that it has no more than \(2\) leaves.

		If \(G\) has at least \(2\) nodes, then \(T\) must have exactly \(2\) leaves, otherwise it means that \(G\) has one node.

		Since a spanning tree is connected, consider the path from one leaf to the other.

		Denote the leaf to leaf path by \(P = (v_1, v_2, \dots, v_n)\). Observe \((v_i, v_{i+1}) \in E\) for \(i = 1, 2, \dots, n -1\)

		But by definition, this is precisely a hamiltonian path.

		Hence, hamiltonian path is a special case of BLST where \(k = 2\).

	\end{soln}

	\question[3] You showed in the previous question that there is an NP-complete problem that is a special case of BLST, and it is not difficult to show that BLST is in NP (though we are not asking you to do this). Does this imply that BLST is NP-complete? Justify your answer.

	\ifsolutions\begin{soln}
	Proof: We provide a proof by induction on the number of nodes, \(n \in \mathbb{N}\), for graph \(G = (V, E)\).

	Base case \(|V| = 1\). The highest degree can be \(d = 0\), since the edge set would be empty.

	Thus, the algorithm uses \(1 = d + 1\) colours to colour the single node and the base case holds.

	Assume for any graph with \(n\) nodes that the algorithm will colour any ordering of nodes using at most \(d + 1\) colours.

	Consider a graph \(G = (V, E)\) with \(|V| = n + 1\) nodes with the highest degree of any node is \(d\).

	Let \(v_1, v_2, \dots, v_n, v_{n+1}\) be any ordering of the nodes. Remove \(v_{n+1}\) from this order, and the graph.

	Denote the deleted graph without \(v_{n+1}\) by \(G'\). Notice the degree of any node in \(G'\) can only decrease.

	Thus, the maximum degree of \(G'\) remains to be \(d\) with the removal of \(v_{n+1}\).

	By assumption, we can colour the ordering \(v_1, v_2, \dots, v_n\) using at most \(d+1\) colours.

	Since \(v_{n+1}\) has degree at most \(d\) then it has at most \(d\) neighbours. Then we add back \(v_{n+1}\).

	The number of colours used to colour its neigbours can be at most \(d\), if each are coloured distinctly.

	Thus, there remains at least \(1\) unique colour to colour \(v_{n+1}\) so that it is a valid colouring.

	In other words, the algorithm uses at most \(d + 1\) colours to colour the ordering \(v_1, v_2, \dots, v_{n+1}\).

	Induction makes the claim holds true for any graph with \(n\) nodes and maximum degree \(d\).


\end{soln}
\fi

	\begin{soln}
		This does not necessarily imply that BLST is NP-complete. We only so far know that BLST is NP.

		To prove BLST is NP-complete, we must show that for every other NP problem \(Y\), that \(Y \leq_p\) BLST.

		While we showed in the previous part that an NP-complete problem (HAMPATH) is a special case of BLST (for \( k = 2 \)), this is not sufficient to conclude NP-completeness of BLST in general.

		For example, we can lax the conditions on HAMPATH and ask for any length path from \(s - t\).

		The generalization becomes, does there exist a path from \(s - t\)? Disregaring needing exactly \(|V|\) nodes.

		We know this is polynomial time solvable using BFS, assuming we have correct adjacency lists.

		Thus, the special case becomes NP-complete, but the general is not. Hence, the issue is lack of reduction.

		In particular, it remains to show that \(BLST(k = 2) \leq_p BLST\) or equivalently \(HAMPATH \leq_p BLST\).

		That is, we must show a polynomial time reduction from HAMPATH, not just observe that one is a special case of the other.

		Hence, unless such a reduction is explicitly given with proof of correctness, we cannot conclude that BLST is NP-complete.
	\end{soln}


	\question[4] Give an example of a polytime-solvable problem you have seen in this class that is a special case of an NP-complete problem. Justify your answer.

	\ifsolutions\begin{soln}
	We consider the graph \(G = (V, E)\) with \(V = \{1, 2, 3, 4\}\) and \(E = \{(1, 4), (4, 3), (3, 2)\}\).

	\begin{center}
		\begin{tikzpicture}[node distance=2cm, auto]
			% Nodes
			\node[circle, draw, above  right = of 3] (1) {1};
			\node[circle, draw, right=of 1] (2) {2};
			\node[circle, draw, right=of 2] (3) {3};
			\node[circle, draw, above right=of 2] (4) {4};

			% Edges
			\draw (1) -- (4) -- (3) -- (2);
		\end{tikzpicture}
	\end{center}

	This graph can be coloured with three colours. Namely we can assign \(A = \{1, 3\}\) \(B = \{2, 4\}\).

	We see that no edges are shared between any vertices in \(A, B\) so this is a valid colouring.

	Now consider the ordering \(1, 2, 3, 4\). We first colour \(1\) blue. And then we consider \(2\).

	There is no edge between \(1, 2\) so we colour \(2\) blue and consider \(3\).

	So, its adjacent vertices have been coloured with blue, then we introduce a new colour for it red.

	Now we finally consider \(4\), its neighbours have been coloured with both red and green, so we must introduce a new colour for it purple.

	We have thus coloured this graph using three colours through the greedy algorithm when we could have used two.


\end{soln}
\fi

	\begin{soln}
		Recall the problem we saw in tutorial 1, where we wish to know if a graph is bipartite.

		A graph \( G = (V, E) \) is \textbf{bipartite} if we can partition the vertices \( V \) into two disjoint sets \( U \) and \( W \) such that no two vertices in \( U \) are connected, and no two vertices in \( W \) are connected.

		Then this problem is indeed a special case of the \( k \)-colouring problem with \( k = 2 \).

		We prove this by showing that \( G \) is bipartite if and only if it is 2-colourable.

		Suppose that \( G \) is 2-colourable. Then there is some function
		\[
			f : V \to \{1, 2\}
		\]
		such that if \( (v, u) \in E \), then \( f(u) \neq f(v) \).

		Define the sets
		\[
			V_1 = \{ v \in V : f(v) = 1 \}, \quad V_2 = \{ v \in V : f(v) = 2 \}.
		\]
		We see \( V_1 \cap V_2 = \varnothing \), otherwise there exists \( v \) such that \( f(v) = 1 \) and \( f(v) = 2 \), a contradiction.

		Now, let \( (u, v) \in V_1 \). Then \( f(u) = f(v) = 1 \), which implies by contrapositive that \((u, v) \notin E\).

		A similar statement holds for \( V_2 \), and thus \( G \) is bipartite.

		Now, suppose \( G \) is bipartite. Then there is \( V_1, V_2 \subseteq V \) so that no two vertices in either set are connected.

		Consider the function
		\[
			f : V \to \{1, 2\} \quad \text{defined by} \quad
			f(v) = \begin{cases}
				1 & \text{if } v \in V_1, \\
				2 & \text{if } v \in V_2.
			\end{cases}
		\]

		Let \( (u, v) \in E \). Since \( G \) is bipartite, \( u \) and \( v \) cannot both be in the same set \( V_1 \) or \( V_2 \).

		Hence, \(f(u) \neq f(v)\), and thus \( G \) is 2-colourable.

		Now consider running a BFS on a graph \( G \) starting from some node \( v \), and colour it 1.

		Then for each traversal step at a vertex \( v \), for every edge \( (v, u) \) we travel along, colour \( u \) with the opposite colour of \( v \) if it is not coloured yet.

		If \( u \) has already been coloured and its colour is the same as \( v \), then report that \( G \) is not bipartite.

		This algorithm runs in \( O(n + m) \) time, since we only add the extra computation of assigning colours to each vertex during the BFS traversal.


	\end{soln}

\end{questions}
