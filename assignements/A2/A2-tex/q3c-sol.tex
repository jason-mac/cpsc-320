\begin{soln}
	We consider the graph \(G = (V, E)\) with \(V = \{1, 2, 3, 4\}\) and \(E = \{(1, 4), (4, 3), (3, 2)\}\).

	\begin{center}
		\begin{tikzpicture}[node distance=2cm, auto]
			% Nodes
			\node[circle, draw, above  right = of 3] (1) {1};
			\node[circle, draw, right=of 1] (2) {2};
			\node[circle, draw, right=of 2] (3) {3};
			\node[circle, draw, above right=of 2] (4) {4};

			% Edges
			\draw (1) -- (4) -- (3) -- (2);
		\end{tikzpicture}
	\end{center}

	This graph can be coloured with three colours. Namely we can assign \(A = \{1, 3\}\) \(B = \{2, 4\}\).

	We see that no edges are shared between any vertices in \(A, B\) so this is a valid colouring.

	Now consider the ordering \(1, 2, 3, 4\). We first colour \(1\) blue. And then we consider \(2\).

	There is no edge between \(1, 2\) so we colour \(2\) blue and consider \(3\).

	So, its adjacent vertices have been coloured with blue, then we introduce a new colour for it red.

	Now we finally consider \(4\), its neighbours have been coloured with both red and green, so we must introduce a new colour for it purple.

	We have thus coloured this graph using three colours through the greedy algorithm when we could have used two.


\end{soln}
