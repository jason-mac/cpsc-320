\begin{soln}
	We prove the lemma by induction on \( i \), where \( i \leq \min(k, l) \), and \( G = (g_1, g_2, \dots, g_k) \), \( O = (o_1, o_2, \dots, o_l) \) are the greedy and optimal solutions respectively, both sorted in increasing order of finish time.

	\textbf{Base Case (\(i = 1\)):}
	The greedy algorithm selects the shift with the earliest finish time among all shifts. Since \( o_1 \) is a valid first choice in some optimal solution, the greedy algorithm must finish no later:
	\[
		f(g_1) \leq f(o_1)
	\]

	\textbf{Inductive Hypothesis:}
	Assume for some \( i \geq 1 \), we have:
	\[
		f(g_j) \leq f(o_j) \quad \text{for all } j \leq i
	\]

	\textbf{Inductive Step (\(i + 1\)):}
	We want to show that \( f(g_{i+1}) \leq f(o_{i+1}) \).

	Let \( t \) be the earliest time after which both greedy and optimal resume selecting shifts — that is, the time immediately after \( f(g_i) \). Let \( S \) be the set of remaining unprocessed shifts that do not overlap with \( g_1, \dots, g_i \).

	By the inductive hypothesis, \( f(g_i) \leq f(o_i) \), so any shift in \( O \) that does not overlap with \( o_1, \dots, o_i \) must also not overlap with \( g_1, \dots, g_i \) — i.e., it is in \( S \). In particular, \( o_{i+1} \in S \).

	Now the greedy algorithm chooses \( g_{i+1} \) as the shift in \( S \) with the **earliest finish time**. Since \( o_{i+1} \in S \), we must have:
	\[
		f(g_{i+1}) \leq f(o_{i+1})
	\]

	Thus, the inductive step holds.

	\textbf{Conclusion:}
	By induction, we have shown that for all \( i \leq \min(k, l) \), it holds that:
	\[
		f(g_i) \leq f(o_i)
	\]

\end{soln}
