\begin{soln}
	Let \(V\) be a vertex set. Add \(v \in V\). Then also add \(v_1, v_2, \dots, v_n\) to \(V\).

	Then for \(v_1, v_2, \dots, v_d\) add \((v, v_i) \in E\) for \(i = 1, 2, \dots, d\).

	Then colour the node \(v\) blue, and colour all other nodes in \(V\) red.

	By construction, the degree of all other nodes that are not \(v\) is \(1\) since it's only connected to \(v\).

	Thus, the degree of \(v\) is the maxmimum degree of the graph, which is \(d\).

	Since no nodes are adjacent except for ones to \(v\). Then the colour red is never shared by adjacent vertices.

	Thus, this graph with maximum degree \(d\) can be coloured with two colours.

\end{soln}
