\section{All-Stars}

Suppose you're organizing a tennis tournament between $n$ players which we simply label as $1,2, \ldots ,n$. Each player competes against every other player, and there are no repeat matches and no tied scores. We model the results of the competition as a directed graph with $n$ vertices and exactly one directed edge between each pair of vertices: $G=(V,E)$, with $V =\{1,2, \ldots ,n\}$ where either $(i,j)) \in E$ (if $i$ defeated $j$), or $(j,i)) \in E$ (if $j$ defeated $i$). Such a graph is called a \emph{tournament}.

You want to determine a way to rank these players, but a challenge here is that cycles \textbf{may} exist in your tournament -- e.g., $i$ may defeat $j$ in a match, and $j$ could defeat $k$, but then we could still have $k$ win a match against $i$; this  would lead to a cycle of length three.

A ranking for the tournament is an ordering $(r_1,r_2, \ldots ,r_n)$ where node $r_1$ is established as the overall rank 1 player. Due to the potential existence of cycles it is challenging to find a fair ranking.


\begin{questions}
	\question[4]  We decide that the top ranked player should
	always be some node whose out-degree in the directed graph is maximum. We motivate this claim by proving that if player $i$ has maximum degree, then for every other player $j$ either $i$ beat $j$, or $i$ beat some player $k$ which beat $j$. Prove this claim.

		{\em Hint: A proof by induction on $n$ works well.}

	\ifsolutions\begin{soln}
	Let \(G = (V, E)\) be a graph such that it is a tournament. Let \(i \in V\) be the node such that it has maximum out-degree.

	We aim to prove that \(\forall j \neq i \in V\) either \(i\) beat \(j\) or \(i\) beat \(k \in V\) who beat \(j\) for any size \(|V| = n \in \mathbb{N}\).

	Base case: \(|V| = 2\). Then if \(i\) has max out-degree then \(E = \{(i, j)\}\). Thus, the claim holds.

	Assume the claim holds for \(|V| = n\), with \(n \in \mathbb{N}\).

	Consider a tournament with \(|V| = n + 1\) nodes. Denote the player who has maximum out-degree by \(i\).

	Then remove \(j \neq i\) from this graph, so that we have \(n\) nodes, denote \(V' = V\setminus \{j\}\), with \(i\) beat \(j\).

	Either \(i\) has the maximum out-degree in this graph or not. We consider each case.

	\begin{itemize}
		\item \text{Case \(i\) has maximum out-degree in \(V'\)}. Then, by assumption for each node other node \(j' \in V'\), either \(i\) beat \(j'\) or \(i\) beat \(k'\) who beat \(j'\).
		      Since \(|V'| = n\).

		      By adding back in \(j\), then the statement holds for \(|V| = n + 1\), since \(i\) beat \(j\) by assumption.

		\item \text{Case \(i\) does not have maximum out-degree in \(V'\)}.

		      So, by removing \(j\), every other node decreased their out-degree by at most one.

		      By assumption \(i\) beat \(j\), so its out-degree must have decreased by one in \(V'\).

		      Thus, there was some node \(k \in V\) such that \(k\) had the same out-degree as \(i\), but \(k\) did beat \(j\).

		      Thus, \(j\) beat \(k\). By assumption for each other \(j' \in V'\), either \(k\) beat \(j'\) or \(k\) beat \(k'\) who beat \(j'\).

		      We now add back \(j\) to the graph. It remains that we need a connection to \(j\), as \(k\) did not beat \(j\).

		      Indeed if \(k\) beats \(i\) who beat \(j\), then assumption holds for \(|V| = n + 1\), using \(k\) in the hypothesis.

		      However, if \(k\) did not beat \(i\) then we consider removing \(i\) from this graph.

		      Since \(k\) lost to \(i\), then the out-degree of \(k\) cannot decrease. So \(k\) must have the max out-degree.

		      This graph being size \(n\), means there is some \(v \in V \setminus \{i\}\) so that \(k\) beats \(v\) and \(v\) beats \(j\).

		      Returning \(i\) to the graph, the statement holds for \(|V| = n + 1\), again using \(k\) in the hypothesis.
	\end{itemize}

	By induction, the statement holds for any tournament with \(|V| = n \in \mathbb{N}\).

\end{soln}
\fi

	\question[2] We decide that a reasonable definition  is suggested by the previous result. A sequence of nodes $(r_1,r_2, \ldots ,r_n)$ is a \emph{valid ranking} if for each $i=1,2, \ldots ,n$ the node $r_i$ has maximum out-degree in the sub-tournament induced by the nodes $r_i,r_{i+1}, \ldots ,r_n$. Give a polynomial-time algorithm which decides if a given permutation of the nodes $(v_1,v_2, \ldots ,v_n)$ is a valid ranking.

	\ifsolutions\begin{soln}
	Let \(P = (v_1, v_2, \dots, v_n)\) be a permutation on \(V\).

	We will assume that we have an adjency matrix to represent the edges in the graph \(M\).

	\begin{algorithmic}[1]
		\Procedure {Valid-Ranking}{P, M}
		\For{each $i = 1, 2, \dots, n - 1$}
		\State store the sum from $j = i$ to $j = n$ of $M(j)$ as $d_i$
		\For{each $j = i + 1, \dots, n - 1$}
		\State store the sum from $k = j$ to $k = n$ of $M(k)$ as $d_k$
		\If{$d_i < d_k$}
		\State end the procedure, report it is not valid
		\EndIf
		\EndFor
		\EndFor
		\State report the permutation as valid
		\EndProcedure
	\end{algorithmic}
	This algorithm is \(O(n^3)\), where \(n = |P|\). In the worse case, we have a valid permutation, and the algorithm does work as follows.

	We have to iterate through the permutation list \(n - 1\) times. Access to the elements will be \(O(1)\) as we assume we are given an array.

	Then for each iteration, we are iterating through the entire adjency matrix, except each iteration we are considering one less node.

	Access to the adjency matrix is \(O(1)\) thus the iteration of the adjency matrix is \(O(i^2)\).

	Then, for each \(i\), the work we are doing is \(i^2\). Thus, summing \(i = 1, 2, \dots, n - 1\) gives us \(O(n^3)\).


\end{soln}
\fi

	\question[2] Describe a tournament of size $n$ which has precisely one valid ranking.

	\ifsolutions\begin{soln}
	Suppose we are given a tournament graph \(T = (V, E)\) with a vertex ordering \(v_1, v_2, \dots, v_n\) such that for all \(i < j\), the directed edge \((v_i, v_j)\) is in \(E\). That is, every vertex points to all vertices that come after it in the order.

	Suppose we have a tournament graph \(T = (V, E)\) such that there is an ordering, \(v_1, v_2, \dots, v_n\) so that there is an edge \((v_i, v_j)\) iff \(i < j\).

	This would create one valid ranking, and this ordering is exactly that ranking.

	Before we prove that statement, we will show it is a tournament in the first place, and that is is valid ranking.

	First, we require that every player has played someone else and only once or in other words, either \((v_i, v_j) \in E\) or \((v_j, v_i) \in E\) but not both.

	Let \(v_i, v_j \in V\) such that \(i \neq j\). Then either \(i < j\) or \(j < i\). Then we can only have one of the edge pairs in \(E\) by assumption.

	Next, we require that \(v_i\) has maximum out-degree in its subtournaments.

	Observe that for any \(v_i\) in a subtournament it's out degree is \(d_i = n - i + 1\), independent of the sub tournament.

	Then for any \(i < j\), we see that \(n - j + 1 < n - i + 1\), but this precisely means that \(d_j < d_i\), thus it is valid ranking.

	Proof: For contradiction, suppose that there were two valid rankings.

	Then it must be some permutation on \(P:= v_1, v_2, \dots, v_n\).

	Then there exists some \(i < j\) in \(P\) such that \(j < i\) in \(P'\). But this means that \(d_j < d_i\) in the subtournament for \(j\) in \(P'\).

	This contradicts that \(P'\) was a valid ranking.
\end{soln}
\fi

	\question[4] One brute force method for
	producing all valid rankings is to consider all permutations of the nodes and then use your algorithm for Part b to check each one if it is a valid ranking. This will run in time $O(n! p(n))$ where $p(n)$ is the runtime of your algorithm in Part b.

	Instead, create a better algorithm which runs in time $O(|Val| poly(n))$ where $Val$ is the set of valid rankings, and $poly(n)$ is a polynomial in $n$. (You do not need to analyze the runtime of your algorithm.)

	\ifsolutions\begin{soln}
	Find the set \(S_0\) such that each node in \(S_0\) has equal out-degree and is the greatest amongst all \(V\).

	Find the set \(S_1\) such that each node in \(S_1\) has equal out-degree and is the greatest amongst all \(V \setminus S_0\).

	Find set \(S_k\) so that each node in \(S_k\) has equal out-degree and is the greatest amongst all \(V \setminus \bigcup_{i=0}^{k-1}S_i\).

	Continue until you have a collection, such that \(\bigcup_{i = 0}^{k} S_i = V\).

	Then create a set of permutation, \(P_0, P_1, \dots, P_k\) such that \(P_i\) contains all possible permuations of ordered tuplets on vertices in \(S_i\).

	Next, create another set containing all ordered tuples of \(r = (p_1, p_2, \dots, p_k)\) where each \(p_i \in P_i\).

	Return \(R = \{\text{all possible } r\}\) as the set all possible valid rankings.

\end{soln}
\fi
\end{questions}
