\begin{soln}
	To sort the the list \(V\) in ascending order in terms of its finish time will take \(O(n\log(n))\) time upfront assuming the list is of size \(n\).

	Then an iteration through the list where we add a shift to the solution as we go is \(O(1)\) time.

	At the time of addition, we also have to delete all intersecting shifts. In this case, we can iterate to the next position in the list until \(f(v_i) < s(v_{k})\) where \(i < k\).

	Finding the next appropiate position is then notice that at each traversal in the list we are either adding a shift or moving onto the next position.

	Thus, this whole traversal takes \(O(n)\). Having already spent \(O(n\log(n))\) to sort, the algorithm is bounded by \(O(n\log(n))\).
\end{soln}













