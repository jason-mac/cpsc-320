\begin{soln}
	First, we sort the list \( V \) of \( n \) shifts in ascending order by their finish times. This takes \( O(n \log n) \) time.

	Then we iterate through the list from left to right and start selecting shifts, once a shift has been selected we add it to the solution set.
	The first shift is always selected.

	Next, we traverse to find the next selected shift in the ordered list that does not intersect with it.

	Intersection can be checked in \(O(1)\) time, by verifying that the selected shifts finish time is after the start time of the current potential candidate shift we are looking at.

	Notice in this process, each shift is visited at most once, either it is selected and added to the solution, or it is skipped due to intersection. Thus, the total traversal takes \( O(n) \) time.

	Combining both steps, the overall runtime of the algorithm is: \(O(n \log n + n) = O(n \log n)\).
\end{soln}













