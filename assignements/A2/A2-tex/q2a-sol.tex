\begin{soln}
	Let \(G = (V, E)\) be a graph such that it is a tournament. Let \(i \in V\) be the node such that it has maximum out-degree.

	We aim to prove that \(\forall j \neq i \in V\) either \(i\) beat \(j\) or \(i\) beat \(k \in V\) who beat \(j\) for any size \(|V| = n \in \mathbb{N}\).

	Base case: \(|V| = 2\). Then if \(i\) has max out-degree then \(E = \{(i, j)\}\). Thus, the claim holds.

	Assume the claim holds for \(|V| = n\), with \(n \in \mathbb{N}\).

	Consider a tournament with \(|V| = n + 1\) nodes. Denote the player who has maximum out-degree by \(i\).

	Then remove \(j \neq i\) from this graph, so that we have \(n\) nodes, denote \(V' = V\setminus \{j\}\), with \(i\) beat \(j\).

	Either \(i\) has the maximum out-degree in this graph or not. We consider each case.

	\begin{itemize}
		\item \text{Case \(i\) has maximum out-degree in \(V'\)}. Then, by assumption for each node other node \(j' \in V'\), either \(i\) beat \(j'\) or \(i\) beat \(k'\) who beat \(j'\).
		      Since \(|V'| = n\).

		      By adding back in \(j\), then the statement holds for \(|V| = n + 1\), since \(i\) beat \(j\) by assumption.

		\item \text{Case \(i\) does not have maximum out-degree in \(V'\)}.

		      So, by removing \(j\), every other node decreased their out-degree by at most one.

		      By assumption \(i\) beat \(j\), so its out-degree must have decreased by one in \(V'\).

		      Thus, there was some node \(k \in V\) such that \(k\) had the same out-degree as \(i\), but \(k\) did beat \(j\).

		      Thus, \(j\) beat \(k\). By assumption for each other \(j' \in V'\), either \(k\) beat \(j'\) or \(k\) beat \(k'\) who beat \(j'\).

		      We now add back \(j\) to the graph. It remains that we need a connection to \(j\), as \(k\) did not beat \(j\).

		      Indeed if \(k\) beats \(i\) who beat \(j\), then assumption holds for \(|V| = n + 1\), using \(k\) in the hypothesis.

		      However, if \(k\) did not beat \(i\) then we consider removing \(i\) from this graph.

		      Since \(k\) lost to \(i\), then the out-degree of \(k\) cannot decrease. So \(k\) must have the max out-degree.

		      This graph being size \(n\), means there is some \(v \in V \setminus \{i\}\) so that \(k\) beats \(v\) and \(v\) beats \(j\).

		      Returning \(i\) to the graph, the statement holds for \(|V| = n + 1\), again using \(k\) in the hypothesis.
	\end{itemize}

	By induction, the statement holds for any tournament with \(|V| = n \in \mathbb{N}\).

\end{soln}
