\begin{soln}
	Let \(G = (V, E)\) be graph such that it has maximum degree \(d\) and there exists \(v \in V\) so that \(\deg(v) < d\).

	Then if by removing \(v\), the maximum degree of the graph does decrease, order vertices \(v_1, v_2, \dots, v_{n-1}, v\).

	If maximum degree does not decrease, then we construct the specific ordering.

	Place \(v\) at the end of the ordering then we consider subgraphs, \(G_1, G_2, \dots, G_k\).

	There exists an ordering of the degree \(d\) nodes such that when placed into the graph, its neighbours will have already been coloured.
	If prove this, its done. but how the fuck do i do this?

	Proof: By removing \(v_n\) with \(\deg(v_n) < d\), either the maximum degree of the graph decreases or not.

	If max degree of the graph decreases to at most \(d - 1\), ordering \(v_1, v_2, \dots, v_{n-1}\) uses at most \(d\) colours.

	Then when colouring \(v_n\), notice it has at most \(d - 1\) neighbours who are each uniquely coloured or not.

	Either case, it has access to \(d\) colours and the algorithm will not need to introduce a new colour.

	Thus, this ordering uses at most \(d\) colours. We then consider the other ordering.

	If removing \(v\) does not decrease \(d\), this means each degree \(d\) node that remains is not adjacent to \(v\).

	Then construct the ordering, \(v_1, v_2, \dots, v, \dots, v_{n-1}, v_n\).

	Where each node behind \(v\) has degree less than \(d\) or is adjacent to \(v\) each node after \(v\) has degree \(d\) and is not adjacent to \(v\).

	Now, by construction, the ordering behind \(v\), will have degree at most \(d - 1\) in the deleted \(v\) graph, thus at that point it will use at most \(d\) colours.

	We know there must exist some node, \(v'\) behind \(v\) such that it was not adjacent to \(v\), otherwise the deleted \(v\) graph would have had its degree decrease.

	Thus, for the first which node that is coloured that is not adjacent to \(v\), the algorithm will use its colour from those available \(d\) colours to colour \(v\).

	Then when considering colouring the nodes after \(v\), we notice that by construction, none of those nodes are adjacent to \(v\).
	And every adjacent node to \(v\) has already been coloured.


\end{soln}
