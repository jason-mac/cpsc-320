\begin{soln}
	Let \(G = (V, E)\) be graph such that it has maximum degree \(d\) and there exists \(v \in V\) so that \(\deg(v) < d\).

	We make a claim that if such a graph is connected, then there must exist some node that is connected to a degree \(d\) node such that it has degree less than \(d\).

	If we assume that every node connected to a degree \(d\) node has degree greater than or equal to \(d\) then every node connected to it has degree \(d\) since we assumed maximum degree of \(d\).
	But then we cannot have the node with degree less than \(d\) be connected to any of the degree \(d\) nodes, otherwise it would then have degree \(d\), thus this graph must be disconnected.
	The contrapositive gives us the first claim.

	First observe if node \(v\) with \(\deg(v) < d\) connected to the degree \(d\) node is removed from the graph, then its neighbours must have their degree decrease in the deleted graph.
	Including the degree \(d\) node.

	Thus, we have guaranteed there will always be a node with degree less than \(d\) since we can decrease the degree of the degree \(d\) node,
	given the existence claim of a node connecting to it. If no such node exists then this implies that the maximum degree of the deleted graph is no longer maximum degree \(d\).

	In other words, continuous deletion of the node with degree less than \(d\) creates another sub-graph with the existence of such a node.

	We then construct the following ordering, \(O\):

	\begin{algorithmic}[1]
		\Procedure {Create Ordering}{$G = (V, E)$}

		\State Set \(O\) to empty ordering.

		\While {$V$ is not empty}

		\If {there exists a degree \(d\) node in \(G\)}

		\State Find $v \in V$ such that $\deg(v) < d$ and it is connected to a degree \(d\)

		\ElsIf {there is no such degree \(d\) node in \(G\)}

		\State Find $v \in V$ such that $\deg(v) < d$

		\EndIf

		\State Add $v$ to the front of $O$

		\State Delete $v$ from $V$

		\State Delete all $e$ from $E$ such that $e$ is incident to $v$

		\EndWhile

		\State return the ordering

		\EndProcedure

	\end{algorithmic}

	Claim: Let \(G = (V, E)\) be a connected graph with maximum degree \(d\). Then the ordering produced from Create Ordering uses at most \(d\) colours when the greedy algorithm is run on it.

	Proof: Let \(O = (v_1, v_2, \dots, v_n)\) be the ordering produced from the proposed algorithm.

	We start with an empty graph \(G' = (V', E')\) of empty edges and vertices. We will in order return the vertices and edges.

	Consider the subgraph \(G'(v_d)\) that contains all the nodes up to \(v_d\) in the ordering and the associated edges between those vertices.

	Now, observe that \(v_d\) has at most \(d - 1\) neighbours by construction of the ordering.

	Then by removing \(v_d\) from this subgraph, the remaining \(d - 1\) nodes clearly have at most \(d - 1\) edges, and thus its maximum degree is \(d - 1\).
	Hence, it will be can coloured with at most \(d\) colours through greedy. Since \(v_d\) has at most \(d - 1\) neighbours, then it can still be coloured uniquely by the other \(d\) colours.

	For each addition from \(v_{d+1}\) to \(v_{n}\), by construction each of them had at most \(d - 1\) neighbours at the time of removal and thus can be uniquely coloured from the other used \(d\) colours.

	This proves the claim.
\end{soln}
