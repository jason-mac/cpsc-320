\begin{soln}
	Proof: We provide a proof by induction on the number of nodes, \(n \in \mathbb{N}\), for graph \(G = (V, E)\).

	Base case \(|V| = 1\). The highest degree can be \(d = 0\), since the edge set would be empty.

	Thus, the algorithm uses \(1 = d + 1\) colours to colour the single node and the base case holds.

	Assume for any graph with \(n\) nodes that the algorithm will colour any ordering of nodes using at most \(d + 1\) colours.

	Consider a graph \(G = (V, E)\) with \(|V| = n + 1\) nodes with the highest degree of any node is \(d\).

	Let \(v_1, v_2, \dots, v_n, v_{n+1}\) be any ordering of the nodes. Remove \(v_{n+1}\) from this order, and the graph.

	Denote the deleted graph without \(v_{n+1}\) by \(G'\). Notice the degree of any node in \(G'\) can only decrease.

	Thus, the maximum degree of \(G'\) remains to be \(d\) with the removal of \(v_{n+1}\).

	By assumption, we can colour the ordering \(v_1, v_2, \dots, v_n\) using at most \(d+1\) colours.

	Since \(v_{n+1}\) has degree at most \(d\) then it has at most \(d\) neighbours. Then we add back \(v_{n+1}\).

	The number of colours used to colour its neigbours can be at most \(d\), if each are coloured distinctly.

	Thus, there remains at least \(1\) unique colour to colour \(v_{n+1}\) so that it is a valid colouring.

	In other words, the algorithm uses at most \(d + 1\) colours to colour the ordering \(v_1, v_2, \dots, v_{n+1}\).

	Induction makes the claim holds true for any graph with \(n\) nodes and maximum degree \(d\).


\end{soln}
