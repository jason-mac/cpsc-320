\section{Colour Me Confused}

Given a graph $G = (V, E)$, we consider the problem of assigning a colour to each vertex in $G$ such that no two adjacent vertices share a colour. Recall that the \emph{degree} of a vertex $v\in V$ is the number of edges incident on $v$. Let $d$ be the \emph{maximum degree} in the graph. 

\begin{center}
\begin{tikzpicture}[node distance=2cm, auto]
  % Nodes
  \node[circle, draw] (1) {1};
  \node[circle, draw, right=of 1] (2) {2};
  \node[circle, draw, right=of 2] (3) {3};
  \node[circle, draw, above right=of 2] (4) {4};

  % Edges
  \draw (1) -- (2) -- (3);
  \draw (2) -- (4);
  \draw (4) -- (3);
\end{tikzpicture}
\end{center}

The maximum degree $d$ in the graph above is $3$. The graph has a 3-colouring (by using different colours for nodes 1, 2, and 3 and colouring node 4 the same as 1 or 2) and a 4-colouring (by using a different colour for every node), but does not have a 2-colouring.

\begin{questions}

\question[2] For any value of $d$, describe a graph with maximum degree $d$ that can be coloured with two colours.

\ifsolutions\begin{soln}
	Let \(V\) be a vertex set. Add \(v \in V\). Then also add \(v_1, v_2, \dots, v_n\) to \(V\).

	Then for \(v_1, v_2, \dots, v_d\) add \((v, v_i) \in E\) for \(i = 1, 2, \dots, d\).

	Then colour the node \(v\) blue, and colour all other nodes in \(V\) red.

	By construction, the degree of all other nodes that are not \(v\) is \(1\) since it's only connected to \(v\).

	Thus, the degree of \(v\) is the maxmimum degree of the graph, which is \(d\).

	Since no nodes are adjacent except for ones to \(v\). Then the colour red is never shared by adjacent vertices.

	Thus, this graph with maximum degree \(d\) can be coloured with two colours.

\end{soln}
\fi 

\question[2] Consider the following greedy algorithm to find a colouring for a graph:

\begin{quote}
    Order vertices arbitrarily. Colour the first vertex with colour 1. Then choose the next vertex $v$ and colour it with the lowest-numbered colour that has not been used on any previously-coloured vertices adjacent to $v$. If all previously-used colours appear on a vertex adjacent to $v$, introduce a new colour and number it, and assign the new colour to vertex $v$.
\end{quote}

Prove that this algorithm uses at most $d+1$ colours.

\ifsolutions\begin{soln}
	Proof: We provide a proof by induction on the number of nodes, \(n \in \mathbb{N}\), for graph \(G = (V, E)\).

	Base case \(|V| = 1\). The highest degree can be \(d = 0\), since the edge set would be empty.

	Thus, the algorithm uses \(1 = d + 1\) colours to colour the single node and the base case holds.

	Assume for any graph with \(n\) nodes that the algorithm will colour any ordering of nodes using at most \(d + 1\) colours.

	Consider a graph \(G = (V, E)\) with \(|V| = n + 1\) nodes with the highest degree of any node is \(d\).

	Let \(v_1, v_2, \dots, v_n, v_{n+1}\) be any ordering of the nodes. Remove \(v_{n+1}\) from this order, and the graph.

	Denote the deleted graph without \(v_{n+1}\) by \(G'\). Notice the degree of any node in \(G'\) can only decrease.

	Thus, the maximum degree of \(G'\) remains to be \(d\) with the removal of \(v_{n+1}\).

	By assumption, we can colour the ordering \(v_1, v_2, \dots, v_n\) using at most \(d+1\) colours.

	Since \(v_{n+1}\) has degree at most \(d\) then it has at most \(d\) neighbours. Then we add back \(v_{n+1}\).

	The number of colours used to colour its neigbours can be at most \(d\), if each are coloured distinctly.

	Thus, there remains at least \(1\) unique colour to colour \(v_{n+1}\) so that it is a valid colouring.

	In other words, the algorithm uses at most \(d + 1\) colours to colour the ordering \(v_1, v_2, \dots, v_{n+1}\).

	Induction makes the claim holds true for any graph with \(n\) nodes and maximum degree \(d\).


\end{soln}
\fi 

\question[2] Give and briefly explain an example where this algorithm does not produce an optimal colouring (i.e., where it uses more colours than is necessary to colour the graph).

\ifsolutions\begin{soln}
	We consider the graph \(G = (V, E)\) with \(V = \{1, 2, 3, 4\}\) and \(E = \{(1, 4), (4, 3), (3, 2)\}\).

	\begin{center}
		\begin{tikzpicture}[node distance=2cm, auto]
			% Nodes
			\node[circle, draw, above  right = of 3] (1) {1};
			\node[circle, draw, right=of 1] (2) {2};
			\node[circle, draw, right=of 2] (3) {3};
			\node[circle, draw, above right=of 2] (4) {4};

			% Edges
			\draw (1) -- (4) -- (3) -- (2);
		\end{tikzpicture}
	\end{center}

	This graph can be coloured with three colours. Namely we can assign \(A = \{1, 3\}\) \(B = \{2, 4\}\).

	We see that no edges are shared between any vertices in \(A, B\) so this is a valid colouring.

	Now consider the ordering \(1, 2, 3, 4\). We first colour \(1\) blue. And then we consider \(2\).

	There is no edge between \(1, 2\) so we colour \(2\) blue and consider \(3\).

	So, its adjacent vertices have been coloured with blue, then we introduce a new colour for it red.

	Now we finally consider \(4\), its neighbours have been coloured with both red and green, so we must introduce a new colour for it purple.

	We have thus coloured this graph using three colours through the greedy algorithm when we could have used two.


\end{soln}
\fi 

\question[5] Now, assume that there is at least one vertex $v$ in $V$ with degree \textbf{less than} $d$.  Design a greedy algorithm that will colour vertices of $G$ with at most $d$ colours. You should proceed by first \textbf{ordering the vertices} in some way, and then assigning colours using the colouring strategy in question 4.2. You should explain why your algorithm uses no more than $d$ colours.

\ifsolutions\begin{soln}
	fuck you
\end{soln}
\fi 

\end{questions}