\section{The Path to Victory}

Suppose we have an undirected graph $G$ that is a \textit{path}: namely, its nodes can be written as $v_1, v_2, \ldots v_n$ with an edge between $v_i$ and $v_j$ if and only if $i$ and $j$ are consecutive numbers. Each node $v_i$ has a positive weight $w_i$. For example, in the following path, the weights are shown as numbers drawn inside the nodes:

\begin{figure*}[tbh!]
	\centering
	\includegraphics[width=0.6\textwidth]{path.pdf}
\end{figure*}

In this problem, we want to find the \textit{maximum independent set}. An independent set is a subset of nodes such that no two of them share and edge, and the maximum independent set is the independent set with largest total weight. For example, in the graph above, the largest-weight independent set is $v_2$ and $v_5$, with total weight 17.

Assume you're given an $n$-dimensional array $V$, where $V[i]$ contains the weight of the node $v_i$ (note that we are assuming 1-based indexing).

\begin{questions}
	\question[3] Give a recurrence $M(i)$ (for $0 \le i \le n$) that defines the weight of the maximum independent set of the first $i$ nodes in $G$.

	\ifsolutions\begin{soln}
	Claim: that
\end{soln}
\fi

	\begin{soln}
		\[
			M(i) =
			\begin{cases}
				\max(M(i - 1), M(i - 2) + V[i]), & i \geq 2 \\
				V[1],                            & i = 1    \\
				0,                               & i < 1
			\end{cases}
		\]
	\end{soln}

	\question[5] Give pseudocode for a recursive or iterative dynamic programming solution to find the weight of the maximum independent set in $G$.

	\ifsolutions\begin{soln}
	Let \(P = (v_1, v_2, \dots, v_n)\) be a permutation on \(V\).

	We will assume that we have an adjency matrix to represent the edges in the graph \(M\).

	\begin{algorithmic}[1]
		\Procedure {Valid-Ranking}{P, M}
		\For{each $i = 1, 2, \dots, n - 1$}
		\State store the sum from $j = i$ to $j = n$ of $M(j)$ as $d_i$
		\For{each $j = i + 1, \dots, n - 1$}
		\State store the sum from $k = j$ to $k = n$ of $M(k)$ as $d_k$
		\If{$d_i < d_k$}
		\State end the procedure, report it is not valid
		\EndIf
		\EndFor
		\EndFor
		\State report the permutation as valid
		\EndProcedure
	\end{algorithmic}
	This algorithm is \(O(n^3)\), where \(n = |P|\). In the worse case, we have a valid permutation, and the algorithm does work as follows.

	We have to iterate through the permutation list \(n - 1\) times. Access to the elements will be \(O(1)\) as we assume we are given an array.

	Then for each iteration, we are iterating through the entire adjency matrix, except each iteration we are considering one less node.

	Access to the adjency matrix is \(O(1)\) thus the iteration of the adjency matrix is \(O(i^2)\).

	Then, for each \(i\), the work we are doing is \(i^2\). Thus, summing \(i = 1, 2, \dots, n - 1\) gives us \(O(n^3)\).


\end{soln}
\fi
	\begin{soln}
		\begin{algorithmic}[1]
			\Procedure{MAX-WEIGHT}{$V, n$}
			\If {$n = 1$}
			\State \Return {$V[1]$}
			\EndIf
			\State Define $M[0..n]$
			\State  $M[0] \gets 0$
			\State  $M[1] \gets V[1]$
			\For {$i = 2, 3, 4, \dots n$}
			\State $M[i] = \max(M[i - 1], M[i-2] + V[i])$
			\EndFor
			\State \Return $M[n]$
			\EndProcedure
		\end{algorithmic}

	\end{soln}

	\question[4] Write a function that takes the table from your dynamic programming solution and the array $V$ and returns the indices (in 1-based indexing) of the actual nodes in the maximum independent set (i.e., write an ``explain'' function like we've done in class).

	\ifsolutions\begin{soln}
	Suppose we are given a tournament graph \(T = (V, E)\) with a vertex ordering \(v_1, v_2, \dots, v_n\) such that for all \(i < j\), the directed edge \((v_i, v_j)\) is in \(E\). That is, every vertex points to all vertices that come after it in the order.

	\textbf{Claim 1:} \(T\) is a tournament.

	For every distinct pair \(v_i, v_j\), either \(i < j\) or \(j < i\), so exactly one of \((v_i, v_j)\) or \((v_j, v_i)\) exists in \(E\), but not both. This satisfies the definition of a tournament: for every pair of distinct vertices, there is exactly one directed edge between them.

	\textbf{Claim 2:} The ordering \(v_1, v_2, \dots, v_n\) is a valid ranking.

	Let \(d_i\) be the out-degree of \(v_i\) in the subtournament induced by \(\{v_i, v_{i+1}, \dots, v_n\}\). Since \(v_i\) points to all vertices that follow it, it has out-degree \(d_i = n - i\). For \(i < j\), \(d_i = n - i > n - j = d_j\), so each player has strictly higher out-degree than those who come after them. Thus, the ordering satisfies the requirement of a valid ranking.

	\textbf{Claim 3:} The valid ranking is unique.

	Suppose for contradiction there is another valid ranking \(P' = v_{\pi(1)}, v_{\pi(2)}, \dots, v_{\pi(n)}\) with \(\pi\) a permutation of \(\{1, \dots, n\}\), and \(P' \ne P\). Then there exist indices \(i < j\) such that \(\pi(i) > \pi(j)\); that is, \(v_{\pi(i)}\) appears before \(v_{\pi(j)}\) in \(P'\), but comes later in the original ordering.

	But then in the subtournament starting at \(v_{\pi(i)}\), the vertex \(v_{\pi(i)}\) must have lower out-degree than \(v_{\pi(j)}\), contradicting the assumption that \(P'\) is a valid ranking. So the valid ranking must be unique.

	Suppose we have a tournament graph \(T = (V, E)\) such that there is an ordering, \(v_1, v_2, \dots, v_n\) so that there is an edge \((v_i, v_j)\) iff \(i < j\).

	This would create one valid ranking, and this ordering is exactly that ranking.

	Before we prove that statement, we will show it is a tournament in the first place, and that is is valid ranking.

	First, we require that every player has played someone else and only once or in other words, either \((v_i, v_j) \in E\) or \((v_j, v_i) \in E\) but not both.

	Let \(v_i, v_j \in V\) such that \(i \neq j\). Then either \(i < j\) or \(j < i\). Then we can only have one of the edge pairs in \(E\) by assumption.

	Next, we require that \(v_i\) has maximum out-degree in its subtournaments.

	Observe that for any \(v_i\) in a subtournament it's out degree is \(d_i = n - i + 1\), independent of the sub tournament.

	Then for any \(i < j\), we see that \(n - j + 1 < n - i + 1\), but this precisely means that \(d_j < d_i\), thus it is valid ranking.

	Proof: For contradiction, suppose that there were two valid rankings.

	Then it must be some permutation on \(P:= v_1, v_2, \dots, v_n\).

	Then there exists some \(i < j\) in \(P\) such that \(j < i\) in \(P'\). But this means that \(d_j < d_i\) in the subtournament for \(j\) in \(P'\).

	This contradicts that \(P'\) was a valid ranking.
\end{soln}
\fi
	\begin{soln}
		\begin{algorithmic}[1]
			\Procedure{explain}{$V, W, n$}
			\State Define $R := \varnothing$
			\State Define $i := n$
			\While{$i > 3$}
			\If {$M[i] == M[i - 2] + V[i]$}
			\State Add $i$ to $R$
			\State $i := i - 1$
			\EndIf
			\State $i := i - 1$
			\EndWhile
			\If{$i = 1$}
			\State Add $1$ to $R$
			\EndIf
			\State \Return $R$
			\EndProcedure
		\end{algorithmic}
	\end{soln}


\end{questions}
