\section{Medieval Matching Problems}

In the 14th century, the Bubonic plague killed between one-third and one-half of Europe's population. With much of the peasantry wiped out by the plague, there was a severe labour shortage as there were no longer enough serfs, tenant farmers, and labourers to work the existing farmland. As a positive result of all this, labourers had more say over whom they would work for, which led to increased wages, better working conditions, and, ultimately, the end of European feudalism.

Here we consider an SMP variant where we have $k$ lords (employers), where employer $i$ has $s_i$ ``slots'' for peasant farmers (applicants), but the total number of slots $\sum_{i=1}^k s_i$ (which we'll denote by $n$) exceeds the number of workers (which we'll call $m$). Each employer has a complete preference list of applicants and vice versa. Moreover, every employer would rather have a slot filled by any applicant than leave a slot unfilled, since leaving a slot unfilled means land will go unmaintained and crops will rot in the fields. We call this the \textbf{Labour Shortage Matching Problem} (LSMP). A solution to LSMP requires that every applicant be matched to an employer, but not that every employer's slot be filled (since that would be impossible).

\begin{questions}
	
	\question[1] We will need to expand our definition of ``instability'' for this problem. The definition we saw in class for RHP still applies to LSMP: namely, when $e_i$ is matched with one or more applicants and $a_j$ is matched with an employer, but $a_j$ prefers $e_i$ to their current employer and $e_i$ prefers $a_j$ over any one of its matched applicants.
	
	Describe a new type of instability that can occur between an applicant and an employer with one or more unfilled slots.
	
	\ifsolutions\begin{soln}
	Let \(G = (V, E)\) be a graph such that it is a tournament. Let \(i \in V\) be the node such that it has maximum out-degree.

	We aim to prove that \(\forall j \neq i \in V\) either \(i\) beat \(j\) or \(i\) beat \(k \in V\) who beat \(j\) for any size \(|V| = n \in \mathbb{N}\).

	Base case: \(|V| = 2\). Then if \(i\) has max out-degree then \(E = \{(i, j)\}\). Thus, the claim holds.

	Assume the claim holds for \(|V| = n\), with \(n \in \mathbb{N}\).

	Consider a tournament with \(|V| = n + 1\) nodes. Denote the player who has maximum out-degree by \(i\).

	Then remove \(j \neq i\) from this graph, so that we have \(n\) nodes, denote \(V' = V\setminus \{j\}\), with \(i\) beat \(j\).

	Either \(i\) has the maximum out-degree in this graph or not. We consider each case.

	\begin{itemize}
		\item \text{Case \(i\) has maximum out-degree in \(V'\)}. Then, by assumption for each node other node \(j' \in V'\), either \(i\) beat \(j'\) or \(i\) beat \(k'\) who beat \(j'\).
		      Since \(|V'| = n\).

		      By adding back in \(j\), then the statement holds for \(|V| = n + 1\), since \(i\) beat \(j\) by assumption.

		\item \text{Case \(i\) does not have maximum out-degree in \(V'\)}.

		      So, by removing \(j\), every other node decreased their out-degree by at most one.

		      By assumption \(i\) beat \(j\), so its out-degree must have decreased by one in \(V'\).

		      Thus, there was some node \(k \in V\) such that \(k\) had the same out-degree as \(i\), but \(k\) did beat \(j\).

		      Thus, \(j\) beat \(k\). By assumption for each other \(j' \in V'\), either \(k\) beat \(j'\) or \(k\) beat \(k'\) who beat \(j'\).

		      We now add back \(j\) to the graph. It remains that we need a connection to \(j\), as \(k\) did not beat \(j\).

		      Indeed if \(k\) beats \(i\) who beat \(j\), then assumption holds for \(|V| = n + 1\), using \(k\) in the hypothesis.

		      However, if \(k\) did not beat \(i\) then we consider removing \(i\) from this graph.

		      Since \(k\) lost to \(i\), then the out-degree of \(k\) cannot decrease. So \(k\) must have the max out-degree.

		      This graph being size \(n\), means there is some \(v \in V \setminus \{i\}\) so that \(k\) beats \(v\) and \(v\) beats \(j\).

		      Returning \(i\) to the graph, the statement holds for \(|V| = n + 1\), again using \(k\) in the hypothesis.
	\end{itemize}

	By induction, the statement holds for any tournament with \(|V| = n \in \mathbb{N}\).

\end{soln}
\fi
	
	\question[1] Give and briefly explain a small example in which the only possible stable matching has an employer with no matches.
	\ifsolutions\begin{soln}
	Let \(P = (v_1, v_2, \dots, v_n)\) be a permutation on \(V\).

	We will assume that we have an adjency matrix to represent the edges in the graph \(M\).

	\begin{algorithmic}[1]
		\Procedure {Valid-Ranking}{P, M}
		\For{each $i = 1, 2, \dots, n - 1$}
		\State store the sum from $j = i$ to $j = n$ of $M(j)$ as $d_i$
		\For{each $j = i + 1, \dots, n - 1$}
		\State store the sum from $k = j$ to $k = n$ of $M(k)$ as $d_k$
		\If{$d_i < d_k$}
		\State end the procedure, report it is not valid
		\EndIf
		\EndFor
		\EndFor
		\State report the permutation as valid
		\EndProcedure
	\end{algorithmic}
	This algorithm is \(O(n^3)\), where \(n = |P|\). In the worse case, we have a valid permutation, and the algorithm does work as follows.

	We have to iterate through the permutation list \(n - 1\) times. Access to the elements will be \(O(1)\) as we assume we are given an array.

	Then for each iteration, we are iterating through the entire adjency matrix, except each iteration we are considering one less node.

	Access to the adjency matrix is \(O(1)\) thus the iteration of the adjency matrix is \(O(i^2)\).

	Then, for each \(i\), the work we are doing is \(i^2\). Thus, summing \(i = 1, 2, \dots, n - 1\) gives us \(O(n^3)\).


\end{soln}
\fi

      \question[2] Describe a (natural) modification of the Gale-Shapley Algorithm which could be  used to solve the Labour Shortage Matching Problem (LSMP). You may use words to explain your algorithm but should also provide pseudocode. No proof of correctness is being requested. Below is the Gale-Shapley algorithm, included mainly so you'll have the LaTeX source to work with:


    \begin{algorithmic}[1]
	   \Procedure{Gale-Shapley}{$E$, $A$}
	   \State Initialize all employers and applicants as unmatched
	     \While {there is an unmatched employer with at least one applicant on its preference list}
	   \State choose such an employer $e \in E$
	   \State make offer to next applicant $a \in A$ on $e$'s preference list
	   \If {$a$ is unmatched}
	   \State Match $e$ with $a$ \Comment{$a$ accepts $e$'s offer}
	   \ElsIf {$a$ prefers $e$ to their current employer $e'$}
	   \State Unmatch $a$ and $e'$ \Comment{$a$ rejects $e'$}
	   \State Match $e$ with $a$ \Comment{$a$ accepts $e$'s offer}
	   \EndIf
	   \State cross $a$ off $e$'s preference list
	   \EndWhile
	   \State report the set of matched pairs as the final matching
	   \EndProcedure
    \end{algorithmic}

      \ifsolutions\begin{soln}
	Suppose we are given a tournament graph \(T = (V, E)\) with a vertex ordering \(v_1, v_2, \dots, v_n\) such that for all \(i < j\), the directed edge \((v_i, v_j)\) is in \(E\). That is, every vertex points to all vertices that come after it in the order.

	Suppose we have a tournament graph \(T = (V, E)\) such that there is an ordering, \(v_1, v_2, \dots, v_n\) so that there is an edge \((v_i, v_j)\) iff \(i < j\).

	This would create one valid ranking, and this ordering is exactly that ranking.

	Before we prove that statement, we will show it is a tournament in the first place, and that is is valid ranking.

	First, we require that every player has played someone else and only once or in other words, either \((v_i, v_j) \in E\) or \((v_j, v_i) \in E\) but not both.

	Let \(v_i, v_j \in V\) such that \(i \neq j\). Then either \(i < j\) or \(j < i\). Then we can only have one of the edge pairs in \(E\) by assumption.

	Next, we require that \(v_i\) has maximum out-degree in its subtournaments.

	Observe that for any \(v_i\) in a subtournament it's out degree is \(d_i = n - i + 1\), independent of the sub tournament.

	Then for any \(i < j\), we see that \(n - j + 1 < n - i + 1\), but this precisely means that \(d_j < d_i\), thus it is valid ranking.

	Proof: For contradiction, suppose that there were two valid rankings.

	Then it must be some permutation on \(P:= v_1, v_2, \dots, v_n\).

	Then there exists some \(i < j\) in \(P\) such that \(j < i\) in \(P'\). But this means that \(d_j < d_i\) in the subtournament for \(j\) in \(P'\).

	This contradicts that \(P'\) was a valid ranking.
\end{soln}
\fi

    \question[3] Give a reduction from LSMP to RHP (recall that RHP is defined in \href{https://canvas.ubc.ca/courses/141582/files/32908753?module_item_id=6761017}{worksheet 2}). Remember that your reduction needs to describe both how to convert an instance of LSMP to an instance of RHP and how to convert the RHP solution back to an LSMP solution.
	\ifsolutions\begin{soln}
	Find the set \(S_0\) such that each node in \(S_0\) has equal out-degree and is the greatest amongst all \(V\).

	Find the set \(S_1\) such that each node in \(S_1\) has equal out-degree and is the greatest amongst all \(V \setminus S_0\).

	Find set \(S_k\) so that each node in \(S_k\) has equal out-degree and is the greatest amongst all \(V \setminus \bigcup_{i=0}^{k-1}S_i\).

	Continue until you have a collection, such that \(\bigcup_{i = 0}^{k} S_i = V\).

	Then create a set of permutation, \(P_0, P_1, \dots, P_k\) such that \(P_i\) contains all possible permuations of ordered tuplets on vertices in \(S_i\).

	Next, create another set containing all ordered tuples of \(r = (p_1, p_2, \dots, p_k)\) where each \(p_i \in P_i\).

	Return \(R = \{\text{all possible } r\}\) as the set all possible valid rankings.

\end{soln}
\fi
	
	\question[4] Prove the correctness of your reduction from 2.4. In other words, prove that if the matching returned by the RHP solver is correct (i.e., is a perfect matching with no instabilities and the correct number of residents at each hospital), then the matching returned by your reduction will:
	\begin{enumerate}[label=(\alph*)]
		\item Have exactly one employer matched to every applicant and no more than $s_i$ applicants matched to employer $i$ (we can think of this as our definition of a ``valid'' solution to this version of the problem); and
		\item Contain no instabilities.
	\end{enumerate}
	\ifsolutions\input{q2e-sol.tex}\fi

 \end{questions}